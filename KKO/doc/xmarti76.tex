%TODO Spustit program vlna
\documentclass[11pt,a4paper,onecolumn,notitlepage]{article}
\usepackage[czech]{babel}
\usepackage[utf8]{inputenc}
\usepackage{lmodern}
\usepackage[T1]{fontenc}
\usepackage[text={17cm, 24cm},left=2cm,top=3cm]{geometry}
\usepackage{graphicx}
\usepackage{enumitem}

% arg
\providecommand{\argv}[2]{\texttt{-{}-#1} (\texttt{-#2})}



\begin{document}
	\begin{center}
		\begin{figure}[hb]
			\centering
			\includegraphics{FIT.pdf}
		\end{figure}
	\vspace{\stretch{0.382}}
	\LARGE
	Kódování a komprese dat (KKO)\\
	\Huge
		Komprese obrazových dat s využitím statického a adaptivního Huffmanova kódování\\
	\vspace{\stretch{0.618}}
	\end{center}

{\Large v~Brně \hfill Václav Martinka\\
	\today \hfill xmarti76}

\pagenumbering{gobble}

\newpage

\pagenumbering{arabic}



\section{Zadání}
	Cílem projektu je v programovacím jazyce C/C++ vytvořit aplikaci pro kompresi šedo tónových 	obrazových dat, kde se uplatní principy statického a adaptivního Huffmanova kódování. Vytvořené řešení musí být spustitelné coby tzv. konzolová aplikace (tedy z příkazového řádku) pod OS Linux v prostředí počítačové sítě FIT VUT v Brně.
	
\section{Rozbor problému}


\section{Překlad a spuštění}
	Program lze přeložit pomocí přiloženého \texttt{Makefile}, která obsahuje následující příkazy:
	\begin{description}
		\item[\texttt{all}] popř. pouze \texttt{make} přeloží program.
		\item[\texttt{doc}] vygeneruje \emph{doxygen} dokumentaci.
		\item[\texttt{zip}] vytvoří archiv se zdrojovými kódy
		\item[\texttt{clean}] smaže soubory vzniklé při překladu.
		\item[\texttt{test}] spustí testovací skript
	\end{description}
	
	Chování programu lze ovlivnit pomocí následujících parametrů:
	\begin{description}
		\item[\argv{help}{h}] Zobrazí nápovědu a ukončí program
		\item[\argv{compression}{c}] Vstupní soubor bude komprimován (výchozí režim)
		\item[\argv{decompression}{d}] Vstupní soubor bude dekomprimován (nelez kombinovat s \texttt{--compression})
		\item[\argv{model}{m}] Aktivuje model pro předzpracování vstupních dat. (Automaticky detekováno při dekompresi.)
		\item[\argv{input {i}<file}] Vstupní soubor.
		\item[\argv{output {o}<file}] Výstupní soubor. Výchozí je \texttt{out.raw}.
		\item[\argv{verbous}{v}] Povolí vypisování informací o průběhu.
		\item[\argv{quiet}{q}] Tichý režim, pouze výpisy na \texttt{stderr}.
		\item[\argv{huffman}{h}\texttt{static}] Použe statické Huffmanovo kódování (výchozí).
		\item[\argv{huffman}{h}\texttt{adaptive}] Použije dynamické Huffmanovo kódování. Nelze kombinovat se statickým kódováním, automaticky detekován při dekompresi.
	\end{description}
		
\section{Závěr}

	

	
\end{document}          
